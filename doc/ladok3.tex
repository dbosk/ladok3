\documentclass[a4paper]{report}
\usepackage{noweb}
% Needed to relax penalty for breaking code chunks across pages, otherwise 
% there might be a lot of space following a code chunk.
\def\nwendcode{\endtrivlist \endgroup}
\let\nwdocspar=\smallbreak

\usepackage[hyphens]{url}
\usepackage{hyperref}
\usepackage{authblk}

\usepackage[utf8]{inputenc}
\usepackage[T1]{fontenc}
\usepackage[swedish,british]{babel}
\usepackage{booktabs}

\usepackage[natbib,style=alphabetic,maxbibnames=99]{biblatex}
\addbibresource{ladok_lib.bib}

\usepackage[inline]{enumitem}
\setlist[enumerate]{label=(\arabic*)}

\usepackage[all]{foreign}
\renewcommand{\foreignfullfont}{}
\renewcommand{\foreignabbrfont}{}

%\usepackage{newclude}
\usepackage{import}

\usepackage[strict]{csquotes}
\usepackage[single]{acro}

\usepackage{subcaption}

\usepackage[noend]{algpseudocode}
\usepackage{xparse}

\let\email\texttt

\usepackage[outputdir=ltxobj]{minted}
\setminted{autogobble,linenos}

\usepackage{pythontex}
\setpythontexoutputdir{.}
\setpythontexworkingdir{..}

\usepackage{fancyvrb}

\usepackage{amsmath}
\usepackage{amssymb}
\usepackage{mathtools}
\usepackage{amsthm}
\usepackage{thmtools}
\usepackage[unq]{unique}
\DeclareMathOperator{\powerset}{\mathcal{P}}

\usepackage[binary-units]{siunitx}

\usepackage{pgf}

\usepackage[capitalize]{cleveref}
\crefalias{question}{item}
\crefname{question}{question}{questions}
\Crefname{question}{Question}{Questions}
\creflabelformat{question}{#2\textup{#1}#3}



\title{%
  A LADOK3 Python API
}
\author{%
  Alexander Baltatzis,
  Daniel Bosk,
  Gerald Q.\ \enquote{Chip} Maguire Jr.
}
\affil{%
  KTH EECS\\
  \texttt{\{alba,dbosk,maguire\}@kth.se}
}

\begin{document}
\maketitle

\vspace*{\fill}
\VerbatimInput{../LICENSE}
\clearpage

\begin{abstract}
  We provide a Python wrapper for the LADOK3 REST API\@.
This wrapper provide a useful object-oriented framework for working with LADOK 
and direct API calls that return the unprocessed JSON data directly from 
LADOK\@.

We also provide a command-line interface (CLI) for the library.
We then construct some useful (shell) scripts using this CLI\@.
These scripts automates the reporting of results from Canvas to LADOK\@.

\end{abstract}
\clearpage

\tableofcontents
\clearpage

@
\chapter{Introduction}

This is the documented source code of the LADOK3 API wrapper for Python, the 
package [[ladok3]].
You can install it by running
\begin{minted}{bash}
pip3 install ladok3
\end{minted}
in the terminal.
You can find a quick reference by running
\begin{minted}[firstnumber=last]{bash}
pydoc ladok3
\end{minted}

We provide the main class [[LadokSession]].
The [[LadokSession]] class acts like \enquote{factories} and will return 
objects representing various LADOK data.
These data objects' classes inherit the [[LadokData]] and [[LadokRemoteData]] 
classes.
Data of the type [[LadokData]] is not expected to change, unlike 
[[LadokRemoteData]], which is.
Objects of type [[LadokRemoteData]] know how to update themselves, \ie fetch 
and refresh their data from LADOK.
When relevant they can also write data to LADOK, \ie update entries such as 
results.

One design criteria is to improve performance.
We do this by caching all factory methods of any [[LadokSession]].
The [[LadokSession]] itself is also designed to be cacheable: if the session to 
LADOK expires, it will automatically reauthenticate to establish a new session.



\part{The library}

\documentclass[a4paper]{report}
\usepackage{noweb}
% Needed to relax penalty for breaking code chunks across pages, otherwise 
% there might be a lot of space following a code chunk.
\def\nwendcode{\endtrivlist \endgroup}
\let\nwdocspar=\smallbreak

\usepackage[hyphens]{url}
\usepackage{hyperref}
\usepackage{authblk}

\usepackage[utf8]{inputenc}
\usepackage[T1]{fontenc}
\usepackage[swedish,british]{babel}
\usepackage{booktabs}

\usepackage[natbib,style=alphabetic,maxbibnames=99]{biblatex}
\addbibresource{ladok_lib.bib}

\usepackage[inline]{enumitem}
\setlist[enumerate]{label=(\arabic*)}

\usepackage[all]{foreign}
\renewcommand{\foreignfullfont}{}
\renewcommand{\foreignabbrfont}{}

%\usepackage{newclude}
\usepackage{import}

\usepackage[strict]{csquotes}
\usepackage[single]{acro}

\usepackage{subcaption}

\usepackage[noend]{algpseudocode}
\usepackage{xparse}

\let\email\texttt

\usepackage[outputdir=ltxobj]{minted}
\setminted{autogobble,linenos}

\usepackage{pythontex}
\setpythontexoutputdir{.}
\setpythontexworkingdir{..}

\usepackage{fancyvrb}

\usepackage{amsmath}
\usepackage{amssymb}
\usepackage{mathtools}
\usepackage{amsthm}
\usepackage{thmtools}
\usepackage[unq]{unique}
\DeclareMathOperator{\powerset}{\mathcal{P}}

\usepackage[binary-units]{siunitx}

\usepackage{pgf}

\usepackage[capitalize]{cleveref}
\crefalias{question}{item}
\crefname{question}{question}{questions}
\Crefname{question}{Question}{Questions}
\creflabelformat{question}{#2\textup{#1}#3}



\title{%
  A LADOK3 Python API
}
\author{%
  Alexander Baltatzis,
  Daniel Bosk,
  Gerald Q.\ \enquote{Chip} Maguire Jr.
}
\affil{%
  KTH EECS\\
  \texttt{\{alba,dbosk,maguire\}@kth.se}
}

\begin{document}
\maketitle

\vspace*{\fill}
\VerbatimInput{../LICENSE}
\clearpage

\begin{abstract}
  We provide a Python wrapper for the LADOK3 REST API\@.
This wrapper provide a useful object-oriented framework for working with LADOK 
and direct API calls that return the unprocessed JSON data directly from 
LADOK\@.

We also provide a command-line interface (CLI) for the library.
We then construct some useful (shell) scripts using this CLI\@.
These scripts automates the reporting of results from Canvas to LADOK\@.

\end{abstract}
\clearpage

\tableofcontents
\clearpage

@
\chapter{Introduction}

This is the documented source code of the LADOK3 API wrapper for Python, the 
package [[ladok3]].
You can install it by running
\begin{minted}{bash}
pip3 install ladok3
\end{minted}
in the terminal.
You can find a quick reference by running
\begin{minted}[firstnumber=last]{bash}
pydoc ladok3
\end{minted}

We provide the main class [[LadokSession]].
The [[LadokSession]] class acts like \enquote{factories} and will return 
objects representing various LADOK data.
These data objects' classes inherit the [[LadokData]] and [[LadokRemoteData]] 
classes.
Data of the type [[LadokData]] is not expected to change, unlike 
[[LadokRemoteData]], which is.
Objects of type [[LadokRemoteData]] know how to update themselves, \ie fetch 
and refresh their data from LADOK.
When relevant they can also write data to LADOK, \ie update entries such as 
results.

One design criteria is to improve performance.
We do this by caching all factory methods of any [[LadokSession]].
The [[LadokSession]] itself is also designed to be cacheable: if the session to 
LADOK expires, it will automatically reauthenticate to establish a new session.



\part{The library}

\documentclass[a4paper]{report}
\usepackage{noweb}
% Needed to relax penalty for breaking code chunks across pages, otherwise 
% there might be a lot of space following a code chunk.
\def\nwendcode{\endtrivlist \endgroup}
\let\nwdocspar=\smallbreak

\usepackage[hyphens]{url}
\usepackage{hyperref}
\usepackage{authblk}

\usepackage[utf8]{inputenc}
\usepackage[T1]{fontenc}
\usepackage[swedish,british]{babel}
\usepackage{booktabs}

\usepackage[natbib,style=alphabetic,maxbibnames=99]{biblatex}
\addbibresource{ladok_lib.bib}

\usepackage[inline]{enumitem}
\setlist[enumerate]{label=(\arabic*)}

\usepackage[all]{foreign}
\renewcommand{\foreignfullfont}{}
\renewcommand{\foreignabbrfont}{}

%\usepackage{newclude}
\usepackage{import}

\usepackage[strict]{csquotes}
\usepackage[single]{acro}

\usepackage{subcaption}

\usepackage[noend]{algpseudocode}
\usepackage{xparse}

\let\email\texttt

\usepackage[outputdir=ltxobj]{minted}
\setminted{autogobble,linenos}

\usepackage{pythontex}
\setpythontexoutputdir{.}
\setpythontexworkingdir{..}

\usepackage{fancyvrb}

\usepackage{amsmath}
\usepackage{amssymb}
\usepackage{mathtools}
\usepackage{amsthm}
\usepackage{thmtools}
\usepackage[unq]{unique}
\DeclareMathOperator{\powerset}{\mathcal{P}}

\usepackage[binary-units]{siunitx}

\usepackage{pgf}

\usepackage[capitalize]{cleveref}
\crefalias{question}{item}
\crefname{question}{question}{questions}
\Crefname{question}{Question}{Questions}
\creflabelformat{question}{#2\textup{#1}#3}



\title{%
  A LADOK3 Python API
}
\author{%
  Alexander Baltatzis,
  Daniel Bosk,
  Gerald Q.\ \enquote{Chip} Maguire Jr.
}
\affil{%
  KTH EECS\\
  \texttt{\{alba,dbosk,maguire\}@kth.se}
}

\begin{document}
\maketitle

\vspace*{\fill}
\VerbatimInput{../LICENSE}
\clearpage

\begin{abstract}
  We provide a Python wrapper for the LADOK3 REST API\@.
This wrapper provide a useful object-oriented framework for working with LADOK 
and direct API calls that return the unprocessed JSON data directly from 
LADOK\@.

We also provide a command-line interface (CLI) for the library.
We then construct some useful (shell) scripts using this CLI\@.
These scripts automates the reporting of results from Canvas to LADOK\@.

\end{abstract}
\clearpage

\tableofcontents
\clearpage

@
\chapter{Introduction}

This is the documented source code of the LADOK3 API wrapper for Python, the 
package [[ladok3]].
You can install it by running
\begin{minted}{bash}
pip3 install ladok3
\end{minted}
in the terminal.
You can find a quick reference by running
\begin{minted}[firstnumber=last]{bash}
pydoc ladok3
\end{minted}

We provide the main class [[LadokSession]].
The [[LadokSession]] class acts like \enquote{factories} and will return 
objects representing various LADOK data.
These data objects' classes inherit the [[LadokData]] and [[LadokRemoteData]] 
classes.
Data of the type [[LadokData]] is not expected to change, unlike 
[[LadokRemoteData]], which is.
Objects of type [[LadokRemoteData]] know how to update themselves, \ie fetch 
and refresh their data from LADOK.
When relevant they can also write data to LADOK, \ie update entries such as 
results.

One design criteria is to improve performance.
We do this by caching all factory methods of any [[LadokSession]].
The [[LadokSession]] itself is also designed to be cacheable: if the session to 
LADOK expires, it will automatically reauthenticate to establish a new session.



\part{The library}

\documentclass[a4paper]{report}
\usepackage{noweb}
% Needed to relax penalty for breaking code chunks across pages, otherwise 
% there might be a lot of space following a code chunk.
\def\nwendcode{\endtrivlist \endgroup}
\let\nwdocspar=\smallbreak

\usepackage[hyphens]{url}
\usepackage{hyperref}
\usepackage{authblk}

\input{preamble.tex}

\title{%
  A LADOK3 Python API
}
\author{%
  Alexander Baltatzis,
  Daniel Bosk,
  Gerald Q.\ \enquote{Chip} Maguire Jr.
}
\affil{%
  KTH EECS\\
  \texttt{\{alba,dbosk,maguire\}@kth.se}
}

\begin{document}
\maketitle

\vspace*{\fill}
\VerbatimInput{../LICENSE}
\clearpage

\begin{abstract}
  \input{abstract.tex}
\end{abstract}
\clearpage

\tableofcontents
\clearpage

@
\chapter{Introduction}

This is the documented source code of the LADOK3 API wrapper for Python, the 
package [[ladok3]].
You can install it by running
\begin{minted}{bash}
pip3 install ladok3
\end{minted}
in the terminal.
You can find a quick reference by running
\begin{minted}[firstnumber=last]{bash}
pydoc ladok3
\end{minted}

We provide the main class [[LadokSession]].
The [[LadokSession]] class acts like \enquote{factories} and will return 
objects representing various LADOK data.
These data objects' classes inherit the [[LadokData]] and [[LadokRemoteData]] 
classes.
Data of the type [[LadokData]] is not expected to change, unlike 
[[LadokRemoteData]], which is.
Objects of type [[LadokRemoteData]] know how to update themselves, \ie fetch 
and refresh their data from LADOK.
When relevant they can also write data to LADOK, \ie update entries such as 
results.

One design criteria is to improve performance.
We do this by caching all factory methods of any [[LadokSession]].
The [[LadokSession]] itself is also designed to be cacheable: if the session to 
LADOK expires, it will automatically reauthenticate to establish a new session.



\part{The library}

\input{../src/ladok3/ladok3.tex}



\part{API calls}

\input{../src/ladok3/api.tex}



\part{A command-line interface}

\chapter{The base interface}

\input{../src/ladok3/cli.tex}


\chapter{The \texttt{data} command}

\input{../src/ladok3/data.tex}



\part{Other example applications}

\chapter{Transfer results from KTH Canvas to LADOK}

Here we provide an example program~[[prgi.py]] which exports results from KTH 
Canvas to LADOK for the introductory programming course~prgi (DD1315).

\input{../examples/prgi.tex}


\printbibliography


\end{document}




\part{API calls}

\input{../src/ladok3/api.tex}



\part{A command-line interface}

\chapter{The base interface}

\input{../src/ladok3/cli.tex}


\chapter{The \texttt{data} command}

\input{../src/ladok3/data.tex}



\part{Other example applications}

\chapter{Transfer results from KTH Canvas to LADOK}

Here we provide an example program~[[prgi.py]] which exports results from KTH 
Canvas to LADOK for the introductory programming course~prgi (DD1315).

\input{../examples/prgi.tex}


\printbibliography


\end{document}




\part{API calls}

\input{../src/ladok3/api.tex}



\part{A command-line interface}

\chapter{The base interface}

\input{../src/ladok3/cli.tex}


\chapter{The \texttt{data} command}

\input{../src/ladok3/data.tex}



\part{Other example applications}

\chapter{Transfer results from KTH Canvas to LADOK}

Here we provide an example program~[[prgi.py]] which exports results from KTH 
Canvas to LADOK for the introductory programming course~prgi (DD1315).

\input{../examples/prgi.tex}


\printbibliography


\end{document}




\part{API calls}

\input{../src/ladok3/api.tex}



\part{A command-line interface}

\chapter{The base interface}

\input{../src/ladok3/cli.tex}


\chapter{The \texttt{data} command}

\input{../src/ladok3/data.tex}



\part{Other example applications}

\chapter{Transfer results from KTH Canvas to LADOK}

Here we provide an example program~[[prgi.py]] which exports results from KTH 
Canvas to LADOK for the introductory programming course~prgi (DD1315).

\input{../examples/prgi.tex}


\printbibliography


\end{document}
